\documentclass[12pt,twocolumn]{IEEEtran}

\usepackage[utf8]{inputenc}
\usepackage{amsmath}
\usepackage{mathtools}
\usepackage{tcolorbox}
\usepackage{graphics}
\title{\textbf{Assignment-4}\\ \large AI1110: Probability And Random Variables\\ IIT Hyderabad}
\author{Pathlavath Shankar (CS21BTECH11064)\\ }

\begin{document}
    \maketitle
    \textbf{Q Papaulis 13.8;}Find the noncausal estimators H1(\omega) \ and \ H2(\omega)\
    \text{respectively of a process s(t) and}\\
   \text{ its derivative s'(t) in terms of the data s(t) = s(t) + v(t)}\\
   \text{ where}\\
   \begin{equation}
       Rs(\tau) = A\frac{sin^2a\tau}{\tau^2}Rv(\tau) = N\delta(\tau) = 0  
   \end{equation}\\
    
    
    \textbf{Solution:-}\textit {The fourier transform Ss(\omega)\ of  the }\\
    \textit{function,} Rs(\tau) = A\frac{sin^2a\tau}{\tau^2}Rv(\tau) \ is\ a \ triangle\\
    \includegraphics[width=0.5\textwidth]{tri.png}\\
    
    \textit{and since Sv(\omega) = N, yields}\\
    
    H1(\omega) = \frac{Ss(\omega)}{Ss(\omega) + Sv(\omega)} \\
    
   = \frac{Aa\pi(1- \omega /2a)}{Aa\pi(1- \omega \2a) + N} \\
   
     \textit{we show next that} H2(\omega) = j\omega H1(\omega).
       
       
      
    
    
\end{document}